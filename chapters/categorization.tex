\chapter{Categorization}
Simple map
Description
\section{symbolic vs subsymbolic}
A first big distinction is to make between symbolic and subsymbolic AI. 
The difference is in how knowledge about the environment gets into an agent. In symbolic AI, knowledge is coded into an agent, while in subsymbolic AI knowledge is generally learned by an agent through observations (=data). 

Which sort of AI is better to use really depends on the circumstances: 
Symbolic AI normally requires a lot more time to program, and needs a programmer well versed in the subject. 
Subsymbolic AI on the other hand depends on Data being available.
Another aspect to keep in mind is the accuracy: Symbolic AI can have a 100 percent accuracy rate when programmed correctly, as it is just logical statements chained together, while with a subsymbolic AI it is normally very hard to get a 100 percent accurate model, and most models are capped at some (high) percentage of accuracy. This is in part because of incorrect/incomplete data, but also because of considerations like time and resource constrains while training.
In general, for easy to implement agents symbolic, and for more complex ones where data is available subsymbolic AI is better suited. In this work I will focus on subsymbolic AI. 
% cite video of backflip
\subsection{Data}
The requirement for data in subsymbolic AI is also the reason why data collection is getting increasingly important in our time. 