\chapter{What is AI?}
\section{Definition}
First it is helpful defining what AI is, so this chapter will start by defining AI.
The field of AI is very broad, and defining this field is quite challenging, but also useful for this thesis. 

To do this, there are a few prerequisites: 
First, the AI has to somehow get a sort of input, like sound from a microphone, click position on a website, ... 
Secondly, the AI has to process this input, 
and lastly it has to have some sort of output, like driving somewhere, or showing different things on a screen, ... 

In cite(modern approach), this whole construct is called an "Agent", the input is called "Percept", and the output "Action".

These agents can act in very different environments.\\This definition is, perhaps not coincidentally, quite similar to how functions might be defined: \\In mathematics, a function is a binary relation between two sets that associates each element of the first set to exactly one element of the second set. 
% cite(https://en.wikipedia.org/wiki/Function_(mathematics)) 
\\In the case of an AI, the first set are the inputs, and the second set are the different outputs. \\Important to note here is, that the input might also be what happened until now, not only what happens at this instance in time. 

These Agents vary widely in their implementation and function, and can range from bots in the web, to your stove. In this thesis however, we will mostly focus on the internal structure of the agents, which can be similar, even in very different environments.
In this thesis the author wants to create an overview of these different internal structures.

As the concept of an agent is so openly defined, it helps to narrow it down further. For example a random number generator has all the prerequisites for an agent, but I will focus only on intelligent agents.

Generally, intelligence is viewed as "the ability to learn, understand and think in a logical way about things; the ability to do this well" 
% cite(https://www.oxfordlearnersdictionaries.com/definition/english/intelligence)
Important to stress here are two seperate concepts: Learning, and thinking in a logical way.
In AI these concepts are sometimes seperated, as to say there is a learning, and a applied phase, where the logical thinking comes into play. \\But there are also models combining these two, where learning takes place while applying the previously learned.

Intelligent agents can still be a lot of things: A calculator has input, processes it intelligently 

Data=Learning/Knowledge=programming symbolisch/subsymbolisch

Mixing