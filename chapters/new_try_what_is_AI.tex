\chapter{What is artificial intelligence?}
% copypaste please
First it is necessary defining what is meant by artificial intelligence in this thesis, as there is a broad range of definitions.%http://agisi.org/Defs_intelligence.html

To get an overview it is helpful to view an artificial intelligence as an entity, a sort of black box. This construct can be called an "Agent". 
This thesis will shed some light into these black boxes.

\section{Defining Artificial intelligence}
\subsection{Agents}
First, an agent has to get an input, like sound from a microphone, or a click position on a website.
Then, it has to process this input, 
and lastly it has to have some sort of output, like driving somewhere, or showing different things on a screen.

\myfigure{figures/Agents.pdf}
    {width=0.8\textwidth, height=0.5\textheight} % max width / height
    {Interaction of an agent and its environment}   % caption
    {Agent}   % optional short caption for table of figures
    {fig:agent}    % label

We call the input "Percept", and the output "Action". \ref{fig:agent}
% cite(modernapproach)

Agents vary widely in their implementation and function, and can range from bots on the web, to the microchip in your smart stove or Roomba. 
There are many kinds of agents, because they have to act in different environments, for example a computer, a room where a robot is driving around, a map, a game, ...

This thesis however, will mostly focus on the internal structure of the agents, which can be similar, even through the corresponding agents are acting in very different environments.
The goal of this thesis is an overview of these different internal structures.

As the concept of an agent is so openly defined, it helps to narrow it down further. For example a card shuffler has all the prerequisites for an agent, but this thesis will focus only on intelligent agents.

For this, we must define intelligence first.
\subsection{Intelligence}
Intelligence can be viewed as "the ability to learn, understand and think logically about things; the ability to do this well" 
% cite(https://www.oxfordlearnersdictionaries.com/definition/english/intelligence)

Important to stress here are two separate concepts: Learning, and thinking logically.
In artificial intelligence, these concepts can be separated, as to say there is a learning, and an applied phase, where the logical thinking comes into play. % examples+more detailed
There are also models combining these two phases, where learning takes place while also applying the previously learned. 


\subsection{Artificial intelligence}
Artificial Intelligence can therefore be defined as an intelligent agent, which this work will focus on. 
% does not work, this is it. this is the part i will write about.
This restriction of only intelligent agents is still not very narrow, as intelligent agents can still be a lot of things: A calculator has input, processes it intelligently, and shows the result, as does a player of a game.

\section{Further definitions}
\subsection{Combination}
But there are very different models for artificial intelligence, as will be explained in more detail in the next chapters, they all have their advantages, but of course their own little problems as well. So in practical use, different models are often chained together, to leverage the advantages of each sort of agent, to create an agent better than the individual agents. %erewrite
A GAN - a generative adverserial network would be one example: In such a setup there are two sub-agents (Generator and Discriminator) working against eachother, as depicted in \ref{fig:GAN}.

\myfigure{figures/GAN.pdf}
    {width=0.8\textwidth, height=0.5\textheight} % max width / height
    {a generative adversarial network}   % caption
    {GAN}   % optional short caption for table of figures
    {fig:GAN}    % label


\subsection{Overfitting}
\label{sec:overfit}
When a model performs good on a trainingset, but bad on the testing data, this is called overfitting. It happens when the model is "memorizing" the answers, which results in less general knowledge.

\subsection{Cross validation}
To combat overfitting cross validation can be used. Cross validation is a process where the border between training and test data is defined differently a few times, and the same learning algorithm is applied to all of these cases. Then the best one is used.