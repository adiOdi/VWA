\chapter{What is AI?}
\section{Definition}
First it is necessary defining what is meant by AI in this thesis, as there is a broad range of definitions.%http://agisi.org/Defs_intelligence.html

To get an overview it is helpful to view an AI as an entity, a sort of black box. This construct can be called an "Agent". 
This thesis will shed some light into these black boxes.

\subsection{Agents}
An agent has to somehow get a sort of input, like sound from a microphone, or a click position on a website.
Then, it has to process this input, 
and lastly it has to have some sort of output, like driving somewhere, or showing different things on a screen.

\myfigure{figures/Agents.pdf}
    {width=0.8\textwidth, height=0.5\textheight} % max width / height
    {Interaction of an agent and its environment}   % caption
    {Agent}   % optional short caption for table of figures
    {fig:agent}    % label

The input can be called "Percept", and the output "Action". % cite(modernapproach)
Image

These Agents vary widely in their implementation and function, and can range from bots on the web, to the microchip in your smart stove or Roomba. 
Agents are this diverse because they have to act in different environments, for example a computer, a room where a robot is driving around, a map, a game, ...

This thesis however, will mostly focus on the internal structure of the agents, which can be similar, even in different environments.
The goal of this thesis is an overview of these different internal structures.

As the concept of an agent is so openly defined, it helps to narrow it down further. For example a random number generator has all the prerequisites for an agent, but this thesis will focus only on intelligent agents.

\subsection{Intelligence}
For this, we must define intelligence first.

Commonly, intelligence is viewed as "the ability to learn, understand and think logically about things; the ability to do this well" 
% cite(https://www.oxfordlearnersdictionaries.com/definition/english/intelligence)

Important to stress here are two separate concepts: Learning, and thinking logically.
In AI, these concepts can be separated, as to say there is a learning, and an applied phase, where the logical thinking comes into play. %This is called supervised learning.
There are also models combining these two phases, where learning takes place while also applying the previously learned. %This is then called unsupervised learning. %example

\section{Conclusion}
This restriction of only intelligent agents is still not very narrow, as intelligent agents can still be a lot of things: A calculator has input, processes it intelligently, and shows the result, as does a player of a game, ...

\section{General}
\subsection{Usage}
There are very different models for AI, as will be explained in more detail in the next chapters, but of course they all have their own little problems but also advantages. So in practical use, different models are often chained together, to leverage the advantages of each sort of agent, to create an agent better than the individual agents. %examples, but also rewrite
\subsection{Overfitting}
When a model performs good on training, but bad on testing data, this is called overfitting. It happens when the model is "memorizing" the answers, which results in less general knowledge.
\subsection{Cross validation}
To combat overfitting cross validation can be used. Cross validation is a process where the border between training and test data is defined differently a few times, and the same learning algorithm is applied to all of these cases. Then the best one is used.