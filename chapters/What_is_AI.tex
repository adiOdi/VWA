\chapter{What is AI?}

\section{Definition}
Defining AI is quite hard, but according to \cite{russellArtificialIntelligenceModern2010}, artificial intelligence can be defined very broadly, and as every agent acting on an environment intelligently, and rationally.

An Agent is defined as an entity being able to perceive inputs, and act according to its percept sequence.\\ 
An environment can be almost everything: e.g. a website, the real world, or something abstract like a map.\\ 
So in contrast to common belief, AI isn't just machine learning and node networks, but rather a much bigger field, containing many subfields.

\section{Similarities with functions}
One might ask why neural nets are this fitting for the task,  The general challenge in this field is finding a function to map these percept sequences to actions the agent can take. \\ 
A possible method for this are neural nets, which can be described as complex functions. Actually there are many more possible functions, which all are possible, but as there are very practical advantages, discussed in a further chapter, neural nets are quite interesting.

\section{Four possible approaches to AI}
There are a few different approaches to AI, but they are generally categorizable into four sectors: 
\begin{itemize}
    \item Thinking humanly
    \item Acting humanly
    \item Thinking rationally
    \item Acting rationally
\end{itemize}
\subsection{Thinking humanly}
According to \cite[page 3]{russellArtificialIntelligenceModern2010}, the first approach has many weaknesses, mainly not knowing how a human thinks.
\subsection{Acting humanly}
The second approach raises the question whether it actually would be desirable
to model an AI to a human, as “The quest for “artificial flight” succeeded when the Wright brothers and others stopped imitating birds and started using wind tunnels and learning about aerodynamics.”\cite[page 3]{russellArtificialIntelligenceModern2010}. 
\subsection{Thinking rationally}
To think rationally means thinking according to the rules of logic, which, in theory is easy, but comes with the problem of defining the starting conditions to an accuracy of 100\% 
\subsection{Acting rationally}