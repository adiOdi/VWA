\chapter{How is it used, and what effects do those use cases have?}

% \section{Grafiken}

% \subsection{Tolle Graphiken}

% \myfigure{figures/deathstar}
%     {width=0.8\textwidth, height=0.5\textheight} % max width / height
%     {Todesstern aus Lego}   % caption
%     {Todesstern aus Lego}   % optional short caption for table of figures
%     {fig:deathstar_lego}    % label

% Die Abbildung \ref{fig:deathstar_lego} zeigt den Todesstern\footnote{RIP.}.
There are many use cases for AI, but the most popular ones are repetitive tasks. As with any tool, one can use AI in both malicious and well-meant applications.
\section{Malice}
Applications for individual benefit of (mostly companies) include (but are not limited to) Facebook \cite{facebookWerbungAufInstagram} and \cite{googleOnlinewerbungLeichtGemacht}, both companies using AI for advertisements, and not benefiting the end-user.
\section{Greater good}
Other use cases through, are benefiting more of the ones using the service. The author, for example, would put google's search-algorithm \cite{googleFunktioniertGoogleSucheSuchalgorithmen} into this category, as it not only makes searching the web faster, but actually enables it, as without a usable search-algorithm using the web as we are now would be impossible.