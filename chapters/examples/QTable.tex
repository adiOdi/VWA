\section{Q-Tables}
\label{sec:QTable}
An example of a Q-Table can be viewed \cite[TicTacToe Example]{c4f}
A Q-Table is a very literal interpretation of the definition of AI: As an AI maps inputs to outputs, a very simple method would just be a table. Implementing such a table is only possible in discrete space, as every state has to have an action assigned to it.

Training a Q-Table is relatively straightforward: There is a score assigned to every State/Action pair, as seen in \autoref{fig:qtable}. Then the agent can learn with reinforcement learning: Should an action lead to a positive reward the score of the previous state-action pair is updated up slightly and vice versa.

\myfigure{figures/Qtable.pdf}
    {width=0.8\textwidth, height=0.5\textheight} % max width / height
    {This Q-Table could be one found in a robot, who knows its position and speed, and can move left and right}   % caption
    {Possible Q-Table for a robot}   % optional short caption for table of figures
    {fig:qtable}    % label

To update the values the function $Q_{new score}=(1-\alpha)*Q_{old score}+\alpha*reward$ is often used.($\alpha=$ learning rate)
In the example seen in \autoref{fig:qtable} this would mean: when the first entry (1|1|5 → 0.5|0.2) is found by the agent, it would probably try moving left. When the reward is 1 for this action, it would update the entry as follows: $\alpha=0.1$\\$Q_{new score}=(1-0.1)*0.5+0.1*1=0.55$, so the entry would now be: 1|1|5 → 0.55|0.2

