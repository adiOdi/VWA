\section{Q-Tables}
\label{sec:QTable}
\subsection{Definition}
A Q-Table is a very literal interpretation of the definition of AI: As an AI maps inputs to outputs, a very simple method without a function in the background would just be a table. To implement such a table is only possible in discrete space, as every state has to have an action assigned to it.

To train a Q-Table is relatively straightforward: There is a score assigned to every State-Action pair. Then the agent can learn with reinforcement learning: Should an action lead to a positive reward the score of the previous state-action pair is updated up slightly and vice versa.

To update the values often the function $Q_{new}=(1-\alpha)*Q_{old}+\alpha*reward$ is used.($\alpha=$ learning rate)

\subsection{Categorization}
Q-Tables can be categorized as subsymbolical, are trained supervised, and are a sort of parameterless algorithm.