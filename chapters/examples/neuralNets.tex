\section{Neural nets} 
\label{sec:NN}
\cite[p727]{MA}
A neural net is a structure where values are passed on through layers of nodes. Each node performs only a relatively simple operation with parameters inherit to each specific node.

First the amount of layers and nodes has to be defined, and to train the neural net, these parameters can be tuned.

As there are normally too many possible values for the parameters not every possible combination can be brute forced. There are different methods for tuning the parameters. Of course these parameters could, in theory, also be filled in by hand, but this is normally not feasible. Two methods to fill them in are back propagation and evolutionary learning.
% explain evolutionary as well

\subsection{Backpropagation}
In backpropagation \ref{fig:backpropagation}, a supervised method, an input is given, and the output is compared to the known correct output. Now the nodes responsible for the biggest error are slightly nudged in the right direction. 

\subsection{Evolutionary learning}
Evolutionary learning is modled, as the name suggests, after natural processes of evolution. It works by first initializing many agents with random parameters. Then the best performing agents are selected, and slightly modified. This circle continiues until sufficient accuracy is achieved.

\myfigure{figures/VWA-Backpropagation.pdf}
    {width=0.8\textwidth, height=0.5\textheight} % max width / height
    {Neural network and backpropagation}   % caption
    {Backpropagation}   % optional short caption for table of figures
    {fig:backpropagation}    % label