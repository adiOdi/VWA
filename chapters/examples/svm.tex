\section{Support Vector Machines}
\label{sec:SVM}
\subsection{K-Nearest-Neighbours}
\cite[p. 783]{MA}
As a parameterless approach, k-Nearest-Neighbours works by saving the entirety of the training data, representing it in an n-dimensional space, and letting the K-Nearest-Neighbors decide how to act. There are only few parameters to control now, and the amount of parameters is no longer strictly bound by the complexity of the problem at hand. Mainly, there remain: the k, the weight of the different dimensions, and the way the neighbors are combined. 
The weight depends on the real-live weight of the dimensions (for example, color could have a lower significance than the size of an object to categorize).
As for the ways in which the neighbors are combined: There are many options here, like, for example, Majority vote, average, weighed average, ...
\myfigure{figures/KNN.pdf}
    {width=0.8\textwidth, height=0.5\textheight} % max width / height
    {K-Nearest-Neighbours}   % caption
    {Two dimensional K-Nearest-Neighbours map}   % optional short caption for table of figures
    {fig:k-nearest-neighbours}    % label

\subsection{Support vector machines}
\cite[p. 744]{MA}
A problem with the K-Nearest-Neighbours type approach is high storage use, as all the training data has to be stored. To combat this, one can use the fact that in most problems the entries close to the border are more helpful than entries in the middle of a decisive cluster.

So as an extension of K-Nearest-Neighbours, Support Vector Machines were developed. They come with most of the strengths of K-Nearest-Neighbours, while eliminating some weaknesses.

In a Support Vector Machine only the few entries along the border are stored, these are called support vectors, as they "hold up" the border. When a decision needs to be made, only the side of the border has to be checked to come to a conclusion. This can also mean a significant increase in efficiency, as not all distances have to be checked like in K-Nearest-Neighbours.

This reduces the problem to an act of fitting a curve to specific points, which is already well-researched. The parameters to choose here are choosen on the basis of how detailed the curve can be, while still not overfitting.

\myfigure{figures/SVM.pdf}
    {width=0.8\textwidth, height=0.5\textheight} % max width / height
    {Support Vector Machine}   % caption
    {Two dimensional Support Vector Machine map}   % optional short caption for table of figures
    {fig:SVM}    % label

\subsection{Usage}
Support Vector Machines are often the first approach, for they are relatively easy to implement, and yield good results in a wide variety of use cases. 

They are relatively simple to view and understand, should there only be few important dimensions.

